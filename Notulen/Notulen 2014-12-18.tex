\documentclass[11pt,twoside,a4paper]{article}
\usepackage[dutch]{babel}
\usepackage{a4wide,times}

%Roy Blokker
%Martin Geertjes
%Rens Hamburger
%Kevin Hill
%Alex Oudsen
%Joran Out
%Elke Salzmann
%Jeroen van Uffelen

\title{Notulen A1}
\author{
Voorzitter: Jeroen van Uffelen\\
Notulist: Elke Salzmann\\
Aanwezigen: Roy Blokker, Rens Hamburger, Kevin Hill, Alex Oudsen, Joran Out\\
Afwezigen: Martin Geertjes\\
}
\date{\today}
\begin{document}
\maketitle

\section{Opening}
09:05

\section{Vaststellen agenda}
-

\section{VHDL code af en gesynthetiseerd!}
Dit is meer een mededeling, maar dat gaat nog niet voor iedereen op. 

\section{Testen}
Roy begint vandaag met testen op de FPGA. We mogen \'e\'en bordje meenemen in de vakantie, aan het eind van de dag besluiten we wie het meeneemt.

\section{WVTTK}
Ook alles wat buiten de chip wordt gebruikt, moet in het volgende verslag opgenomen worden.

\section{Eindvergadering}

\subsection{Feedback midterm}
\begin{itemize}
\item Datum op voorpagina
\item Samenvatting voor inhoudsopgave
\item Uitgangen beter uitleggen (DCF\_debug)
\item Specificaties zijn te beknopt, zou alles moeten beschrijven
\item Misschien een plaatjes maken van het algemene systeem
\item Uitgangen in blokdiagram beter afstemmen op de rest
\item Niet te diep ingaan op specifieke signalen in systeem overview
\item DCF: Beter uitleggen hoe het signaal gesynchroniseerd wordt
\item Main: captions boven tabel
\item Onder het kopje `Functionaliteit' een stukje tekst voor de subsection
\item Beter de signalen met elkaar afstemmen en de signalen beter uitleggen
\item Alarm: captions bij de figuren
\item Alarm: refereren naar figuren
\item Testen: puur het testen van de chip
\item Welke interne signalen kunnen nog bekeken worden als de uitgangen niet blijken te kloppen
\item Voortgang: Moduleopdracht hoeft er niet per se in
\item Appendix: netjes gedaan met nummering en kleurtjes
\item when others states!!!
\item Main misschien nog splitsen in kleinere modules
\item entity buffer: knoppen en knopjes
\item Simulaties: beter leesbaar (kleuren inverteren)
\item Alarm: nieuw proces aanmaken ipv if'jes en else'jes
\item Appendices ordenen (bijv, A: VHDL, B: testbenches, C:simulaties)
\item Simulaties: alleen behaviour en switch level
\item Indicatie van grootte chip
\end{itemize}

LCD moet nog ingeleverd worden.
\section{Rondvraag}

\section{Sluiting}

\end{document}
