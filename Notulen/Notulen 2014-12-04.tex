\documentclass[11pt,twoside,a4paper]{article}
\usepackage[dutch]{babel}
\usepackage{a4wide,times}

%Roy Blokker
%Martin Geertjes
%Rens Hamburger
%Kevin Hill
%Alex Oudsen
%Joran Out
%Elke Salzmann
%Jeroen van Uffelen

\title{Notulen A1}
\author{
Voorzitter: Jeroen van Uffelen\\
Notulist: Elke Salzmann\\
Aanwezigen: Roy Blokker, Martin Geertjes, Rens Hamburger, Kevin Hill, Alex Oudsen, Joran Out\\
Afwezigen: \\
}
\date{\today}
\begin{document}
\maketitle

\section{Opening}
9:05

\section{Vaststellen agenda}
-

\section{Plan van Aanpak}
Bijna af, is aan het eind van de dag af.

\section{Afspraken VHDL, projectverslag}
Belangrijkste afspraken voor VHDL: De code moet voorzien zijn van goed commentaar en mag geen hoofdletters bevatten.\\
In de handleiding staan in hoofdstuk 7 de eisen voor de VHDL-code. Degene die dat nog niet gelezen hebben, moeten dat nog doorlezen.

\section{Bespreken (sub)onderdelen en kortsluiten van aansluitingen}
Jeroen en Martin willen ook de dag en datum weergeven op de LCD. Maar als dat uit de DCF controller komt, moeten er veel draden over de chip. Eventueel kan het register van de dagen en datum in de LCD controller geïmplementeerd worden.\\
Om te zorgen dat er maar \'e\'en puls doorgegeven wordt als er op een knop gedrukt wordt, moet er een soort inputbuffer geimplementeerd worden.\\
We gaan de tijd van de wekker implementeren en later uitzoeken hoe we het probleem op kunnen lossen dat het alarm niet om twaalf uur gezet kan worden.\\
Het LCD-scherm kan ook op een klok van 32 kHz werken, dus we gaan met maar \'e\'en kloksignaal werken.\\
Er moet een afspraak gemaakt worden over hoe het signaal van de uren en minuten van de DCF naar de LCD.

\section{WVTTK}
-

\section{Rondvraag}
-

\section{Schorsing}
9:44

\section{Heropening}
12:15

\section{PvA}
Vandaag wordt het opgestuurd voor zes uur naar de rest van de projectgroep. Morgen wordt de definitieve versie opgestuurd.

\section{Planning}
Ontwerp is tot en met volgende week. Dus donderdag 11 december moet alle VHDL code af zijn.\\
17 december moet het mid-term verslag ingeleverd worden.

\section{Git}
git clone\\
git add (-A)\\
git commit -m\\
git pull\\
git push\\
git stash (ga terug naar de laatste commit)

\end{document}
