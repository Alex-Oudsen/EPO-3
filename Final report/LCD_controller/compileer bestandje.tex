\documentclass[a4paper, 10 pt]{report}
\usepackage[dutch]{babel}
\usepackage[latin1]{inputenc}
\usepackage{pdfpages}
\usepackage{graphicx}
\usepackage{caption}
\usepackage{subcaption}
\usepackage{listings}     % Bron codes toe te voegen
\usepackage{tabularx}
\usepackage{hyperref}


\title{Module SPI Controller}

\author{Martin Geertjes (4324285), Jeroen van Uffelen (4232690)}


\begin{document}
\maketitle
\newpage
\title{\textbf{Samenvatting}}\\
bladiebla

\newpage


\chapter{inleiding}
blaballa..\\

\newpage 

%\chapter{LCD controller}
\chapter{Inleiding}
Op de LCD zal de huidige tijd, ingestelde wekkertijd, datum en ingeschakelde functies te zien zijn. Een LCD is daar handig voor omdat het veel ontwerp vrijheid bied. Dat neemt ook 
\section{Specificaties}

\section{Entity}

\begin{center}
\label{table:uitgangen}
\begin{tabular}{| l | l | p{4.5cm} |}
\hline
\textbf{Naam} & \textbf{Type} & \textbf{Functie} \\ \hline
clk           				& in std$\_$logic           									& Klok             \\ \hline
reset         			& in std$\_$logic           									& Reset            \\ \hline
ready					& in std$\_$logic 											&  \\ \hline
uren						&in std$\_$logic$\_$vector(5 downto 0) 		&data signaal met actuele uren afkomstig van DCF \\ \hline
minuten				&in std$\_$logic$\_$vector(6 downto 0) 		&data signaal met actuele minuten afkomstig van DCF \\ \hline
dagvdweek			&in std$\_$logic$\_$vector(2 downto 0)			&data signaal met de actuele dag afkomstig van DCF\\ \hline
dagvdmaand		&in std$\_$logic$\_$vector(5 downto 0) 		&data signaal met de actuele dag van de maand afkomstig van DCF \\ \hline
maand					&in std$\_$logic$\_$vector(4 downto 0) 		&data signaal met de actuele maand afkomstig van DCF  \\ \hline
jaar						&in std$\_$logic$\_$vector(7 downto 0)	 		&data signaal met het actuele jaar afkomstig van DCF  \\ \hline
dcf$\_$debug		&in std$\_$logic 												& ????  \\ \hline
menu					&in std$\_$logic$\_$vector(2 downto 0)			&data signaal die de actuele menu state weergeeft \\ \hline
alarm 					&in std$\_$logic												&buffer signaal dat weergeeft of alarmfunctie in of uitgeschakeld is\\ \hline
geluid$\_$signaal 		&in std$\_$logic 		 										&buffer signaal dat weergeeft of geluidsfunctie in of uitgeschakeld is \\ \hline
licht$\_$signaal 	&in std$\_$logic 		 										&buffer signaal dat weergeeft of lichtfunctie in of uitgeschakeld is  \\ \hline 
wektijd$\_$uren 	& in std$\_$logic$\_$vector(5 downto 0)		&data signaal met ingestelde wektijd uren \\ \hline 
wektijd$\_$min 	&in std$\_$logic$\_$vector(6 downto 0) 		&data signaal met ingestelde wektijd minuten \\ \hline 
data$\_$out 		&in std$\_$logic$\_$vector(6 downto 0) 		&data signaal dat de x,y,c informatie doorgeeft aan de microcontroller \\ \hline 
clk$\_$out 			&in std$\_$logic 		 										&clock om microcontroller clock mee te synchroniseren \\ \hline 
\end{tabular}
\end{center}

\subsection{Gedrag}

\section{Functionaliteit}


\section{Subsystemen LCD}
\subsection{...}
\include{bladiebla}

- gedrag
- functionaliteit
- FSM
- VHDL code 
- Testen 
- Simulatie
- Resultaten
- Discussie

	

\subsection{FSM}


\subsection{VHDL code}

\section{Testen}

\section{Simulatie}

\section{Resultaten}

\subsection{Conclusie en discussie}



\
\newpage
\section{Bibliografie}
\begin{thebibliography}{10}
	\bibitem{VHDL_book}	Stephen Brown, Zvonko Vranesic\'c , 							\emph{Fundamentals of Digital Logic with VHDL design}, McGraw-Hill, Jan 1, 2009	
	\bibitem{ds_book}	Jan M. Rabeay, Anantha Chandrakasan, Borivoje Nikoli\'c , 							\emph{Digital Integrated Circuits, second edition}, Prentice Hall, 2003
	
\bibitem{spi_wiki}	Wikipedia\'c , \emph{Serial Peripheral Interface} Geraadpleegd op 10 november 2014, \url{http://nl.wikipedia.org/wiki/Serial_Peripheral_Interface}

\end{thebibliography}


%\section{Broncode spi_master}\label{VHDL:spi_master}
%	\scriptsize
%	\lstinputlisting [language= VHDL] {spi_master.vhd}
%	\normalsize


%\begin[language=VHDL]{lstlisting}
%\input{appendix/spi_master.vhd}
%\end{lstlisting}





\end{document}
