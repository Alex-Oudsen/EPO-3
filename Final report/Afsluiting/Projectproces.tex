\chapter{Voortgang van het project}

\section{Inleiding}
Bij dit project zijn er vaak weinig resultaten, totdat het bijna afgelopen is. 
Dit is een van de redenen dat voor een wake-up light gekozen is.
Een wekker zelf is relatief makkelijk te maken. 
Er zijn echter ook een hele hoop extra features die in een wekker geimplementeerd kunnen worden. 
Op deze manier is dus een werkend resultaat relatief snel geproduceerd, en kunnen daarna naar gelang extra toepassingen toegevoegd worden. 
Dit is goed voor het moreel in de groep, aangezien een werkend product al heel snel gerealiseerd is.
Hierdoor is er ook meer aansporing om meer toepassingen te implementeren, omdat er al een werkend geheel is. 
In het ergste geval is er geen extra feature.
Daarnaast, als uiteindelijk bleek dat de planning te krap was, kunnen er features geschrapt worden, en is er nog steeds een werkend product. 
Onder andere dit maakt een wake-up light zeer aantrekkelijk om te maken.

\section{Werkverdeling}
\subsection{Module opdracht}
De eerste twee weken werd er gewerkt aan een module-opdracht, dit bereide iedereen voor op de echte opdracht. 
De module-opdracht was vergelijkbaar met de uiteindelijke opdracht, alleen veel kleiner en het onderwerp was anders.
De module-opdracht werd in tweetallen voltooid. De onderdelen die gemaakt werden zijn:
\begin{itemize}[nolistsep]
	\item Een ALU
	\item Een SRAM-module
	\item Een FIFO-module
	\item Een SPI-interface
\end{itemize}
Deze zijn allen ter voorbereiding op de grote opdracht. Deze opdrachten zijn in 2 weken tijd voltooid. 
Daarnaast heeft elk tweetal de specificaties voor een ander groepje opgesteld, zodat ook hierin ervaring opgedaan zou worden.
Dit is nodig aangezien voor de grote opdracht zelf de specificaties opgesteld moesten worden. 

\subsection{Wake-up light}
Zodra vastgesteld was wat de grote opdracht zou worden zijn eerst precieze specificaties opgesteld. 
Dit was nodig zodat het duidelijk was wat er gedaan moest worden.
Vervolgens zijn de taken zo snel mogelijk verdeeld door de wake-up light in blokken te verdelen. 
Van deze blokken werden eerst de specificaties bepaald, zodat er geen communicatieproblemen zouden ontstaan tussen de blokken.
Uiteindelijk zijn er 4 hoofdblokken ontstaan, wat goed uitkwam, aangezien dit betekende dat er weer 4 tweetallen nodig waren per blok. \\
Deze blokken werden vervolgens door de tweetallen apart gemaakt, en waar de specificaties niet duidelijk genoeg waren, of onhandig gedefinieerd, werden deze aangepast.

\section{Samenwerking binnen de groep}
5 weken is een zeer korte tijd om mensen te leren kennen. Het merendeel van de samenwerking verliep goed, dit onderdeel zal uitgebreid worden in de loop van de komende 5 weken.

\section{Afspraken binnen de groep}
Afspraken binnen de groep verliepen soepel. De enkele keer dat dit niet gebeurde was hier een goede reden voor. Zo gebeurde het dat op de dag van een presentatie bleek dat een van de leden ziek was. Andere leden sprongen in en zo kwam de presentatie toch nog tot een goed einde.