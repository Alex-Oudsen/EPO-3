\chapter{Voortgang van het project}

\section{Inleiding}
Bij dit project zijn er vaak weinig resultaten, totdat het bijna afgelopen is. 
Dit is een van de redenen dat voor een wake-up light gekozen is.
Een wekker zelf is relatief makkelijk te maken. 
Er zijn echter ook een hele hoop extra features die in een wekker ge\"implementeerd kunnen worden. 
Op deze manier is dus een werkend resultaat relatief snel geproduceerd en kunnen daarna naar gelang extra toepassingen toegevoegd worden. 
Dit is goed voor het moreel in de groep, aangezien een werkend product al heel snel gerealiseerd is.
Hierdoor is er ook meer aansporing om meer toepassingen te implementeren, omdat er al een werkend geheel is. 
In het ergste geval is er geen extra feature.
Daarnaast, als uiteindelijk bleek dat de planning te krap was, kunnen er features geschrapt worden, en is er nog steeds een werkend product. 
Onder andere dit maakt een wake-up light zeer aantrekkelijk om te maken.

\section{Werkverdeling}
Zodra duidelijk was wat gemaakt ging worden, werd een werkplan opgesteld.
Dit was nodig zodat het duidelijk was wat er gedaan moest worden.
Vervolgens zijn de taken zo snel mogelijk verdeeld door de wake-up light in blokken te verdelen. 
Van deze blokken werden ook eerst de specificaties bepaald, zodat er geen communicatieproblemen zouden ontstaan tussen de blokken.
Uiteindelijk zijn er 4 hoofdblokken ontstaan, wat goed uitkwam, aangezien dit betekende dat er weer 4 tweetallen nodig waren, en de groep bestaat uit 8 mensen. \\
Deze blokken werden vervolgens door de tweetallen apart gemaakt en waar de specificaties niet duidelijk genoeg waren, of onhandig gedefinieerd, werden deze aangepast. 


\section{Samenwerking binnen de groep}
De samenwerking binnen de groep was goed. Als afspraken niet duidelijk waren werd snel contact gezocht, hiertoe was onder andere een Whatsapp-groepsgesprek aangemaakt. Van de vier onderdelen van het wake-up light waren er drie ruim op tijd af. De laatste, het LCD-scherm, liet iets langer op zich wachten. Toen de uiteindelijke deadline dichterbij kwam, werd er echter wel uitstekend samengewerkt om het toch nog af te krijgen.


\section{Afspraken en deadlines binnen de groep}
Afspraken binnen de groep verliepen meestal soepel. Zo gebeurde het dat op de dag van een presentatie bleek dat een van de leden ziek was. Andere leden sprongen in en zo kwam de presentatie toch nog tot een goed einde. 