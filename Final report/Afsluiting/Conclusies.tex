\chapter{Conclusie}

De wekker werkt nu naar behoren. Alle onderdelen voldoen aan de gestelde eisen. De interne klok wordt gesynchroniseert als een goed dcf-signaal wordt ontvangen en kan op zichzelf door tellen mocht het signaal wegvallen. De data wordt ook ververst en doorgegeven mits de parity check het signaal goedkeurd. De controller beschikt over een werkend menu en kan bepalen welke informatie op het LCD-scherm weergegeven wordt. Dit kan via de 4 knoppen menu, set, up en down. Daarnaast is de wekker in te stellen via het menu en kan het geluid apart aan of uit gezet worden. Het licht begint 15 minuten van te voren langzaam meer licht te geven en op de ingestelde wekkertijd gaat het geluid af.  De LCD\_controller geeft overzichtelijk alle informatie via een microcontroller weer op een LCD scherm.
Alle onderdelen zijn succesvol gesimuleerd in modelsim\textregistered en via een switch-level simulatie. Nada deze simulaties overeen kwamen zijn alle onderdelen op een FPGA getest. Dit bleek ook succesvol te zijn. Vervolgens werd met succes deze handelingen voor een top-entity herhaald.
\#\#Helaas bleek dat er na al deze handelingen te weinig tijd was om de layout aan de bond bars aan te passen. Zodoende is het niet gelukt om de code in te leveren en zal het product niet op een chip verschijnen.\#\#
\\
Hoewel de wekker theoretisch op nu op de chip kan staan, zal het moeten blijken of de chip het in praktijk ook doet. Naast deze test zijn er nog meer mogelijkheden om te onderzoeken. Zou zouden er meer functionaliteiten en of opties op de wekker kunnen. Zoals bijvoorbeeld het instellen van de duur dat het licht aangaat of de temperatuur weer te geven. Bij zo'n eventuele uitbreiding zal misschien overwogen moeten worden een grotere chip te nemen zodat er meer oppast of meer op effici\"entie letten.