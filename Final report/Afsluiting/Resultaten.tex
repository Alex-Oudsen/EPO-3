\chapter{Results for total design}
Alle subblokken moeten in een top-entity aan elkaar geknoopt worden. Deze code is te vinden in appendix \ref{code:top-level-entity}. Om dit te testen is er een simulatie gedaan. Het resultaat daarvan is te vinden in figuur \ref{fig:uiteinres}.

\begin{figure}[h!]
\includegraphics[width=\textwidth]{behaviour}
\caption{Simulatie resultaat}
\label{fig:uiteinres}
\end{figure}

Hieruit blijkt dat de chip voldoet aan de gestelde eisen. Om een beter beeld te krijgen of deze resultaten eruit komen als de chip daadwerkelijk ontwikkeld is, is er ook de gesynthetiseerde code gesimuleerd. Het resultaat daarvan is te vinden in figuur \ref{fig:uiteinsimres}.

\begin{figure}[h!]
\includegraphics[width=\textwidth]{synthesised.png}
\caption{Gesynthetiseerd resultaat}
\label{fig:uiteinsimres}
\end{figure}

Na deze te hebben vergeleken, bleken die goed overeen te komen.