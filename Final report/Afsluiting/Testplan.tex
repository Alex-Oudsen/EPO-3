\chapter{Plan voor het testen van de chip}

\section{Logic analyzer}
Zodra de chip daadwerkelijk gemaakt is, kan die in kwartaal vier getest worden. Dit wordt gedaan met behulp van de logic analyzer LA-5580. Dit meetinstrument meet de uitgangsspanningen (logische \'e\'en of nul) als functie van de tijd. De voordelen van deze analyzer ten opzichte van een oscilloscoop is dat de analyzer een groot aantal uitgangsspanningen kan meten en tegelijkertijd ingangssignalen de chip in kan sturen.\\
Voor het testen van de chip met een logic analyzer zijn een paar referentiedocumenten nodig die geproduceerd worden tijdens het ontwerpen van de chip. De ingangssignalen die in de referentiedocumenten staan worden als ingang naar de chip gestuurd en de uitgangssignalen van de chip en die van de documenten worden met elkaar vergeleken.

\section{Testsignalen}
Er hadden nog testsignalen op de lege pinnen gezet kunnen worden. Op die manier zouden ook signalen getest kunnen worden die van het ene subblok naar het andere subblok gaan. Mocht er een subblok niet functioneren, kan dat getest worden met behulp van de testsignalen en eventueel kunnen die testsignalen nog gebruikt worden om op een andere manier een bruikbaar en visueel uitgangssignaal te krijgen.\\
Het zou dan om de volgende signalen gaan
\begin{itemize}
\item Het DCF\_debug signaal
\item Het Hz\_1 signaal
\item En de vector menu\_state.
\end{itemize}
Door tijd en ruimtegebrek was dit alleen niet meer mogelijk.