\chapter{Plan voor het testen van de chip}

\section{Logic analyzer}
Zodra de chip daadwerkelijk gemaakt is, kan die in kwartaal vier getest worden. Dit wordt gedaan met behulp van de logic analyzer LA-5580. Dit meetinstrument meet de uitgangsspanningen (logische \'e\'en of nul) als functie van de tijd. De voordelen van deze analyzer ten opzichte van een oscilloscoop is dat de analyzer een groot aantal uitgangsspanningen kan meten en tegelijkertijd ingangssignalen de chip in kan sturen.\\
Voor het testen van de chip met een logic analyzer zijn een paar referentiedocumenten nodig die geproduceerd worden tijdens het ontwerpen van de chip. De ingangssignalen die in de referentiedocumenten staan worden als ingang naar de chip gestuurd en de uitgangssignalen van de chip en die van de documenten worden met elkaar vergeleken.

\section{Testsignalen}
De meeste uitgangssignalen die uit de chip komen, gaan naar een LCD-scherm. Mocht er iets mis gegaan zijn tussen de chip, microcontroller en het scherm, dan zal de weergave op het scherm niet goed zijn. Om te checken of de fout in de chip zelf zit of in de communicatie tussen de chip en het LCD zullen er testsignalen op de niet-gebruikte pinnen van de chip gezet worden. Op die manier kunnen de signalen tussen de subblokken ook getest worden.
De testsignalen die op de pinnen gezet worden zijn de volgende:
\begin{itemize}
\item Het dcf\_debug signaal, om te zien of het DCF-signaal opgevangen wordt.
\item Het \'e\'en-Hertz-signaal
\end{itemize}