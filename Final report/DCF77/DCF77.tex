\chapter{DCF controller}
\section{Inleiding}
In dit onderdeel, genaamd DCF77,  wordt de basis van de wekker gelegd, door verschillende belangrijke datasignalen aan te maken, welke nodig zijn om de rest van de wekker goed te laten functioneren. Een eis die is gesteld aan de eigenschappen van deze klok, is dat deze gesynchroniseerd wordt met het zogenaamde DCF77 signaal. Dit is een signaal dat vanuit Duitsland wordt verzonden en bestaat uit korte (100 ms) en lange (200 ms) pulsen, welke respectievelijk nullen en enen coderen. Iedere seconde, behalve de laatste van iedere minuut, wordt er een puls verzonden. De bits die door deze pulsen worden gecodeerd, bevatten allerlei informatie, zoals de actuele datum en tijd op de eerstvolgende minuut. Een deel van de informatie die op deze manier wordt verzonden, zal worden gebruikt voor het aansturen van de wekker. Om gebruik te kunnen maken van de informatie die met het DCF77 signaal wordt verzonden, is het echter wel nodig te weten welke bit uit de reeks van 59 stuks welke informatie codeert. In figuur \ref{fig:dcfsignaal} is te zien welke informatie door elk van de 59 bits wordt gecodeerd. Een versie in tabelvorm is te vinden op Wikipedia ~\cite{Tijdwiki}. De wekker zal gebruik gaan maken van bits 21 t/m 58. In de afbeelding worden binnen deze selectie bits 28, 35 en 58 respectievelijk P1, P2 en P3 genoemd. Deze bits zijn zogenaamde parity-bits, welke de ontvanger van het DCF77 signaal in staat stelt om tot op zekere hoogte te controleren of de ontvangen bitreeks correct is. In het DCF77 signaal wordt gebruik gemaakt van even parity. Dit betekent dat, wanneer zich in de bits die bij een zekere parity bit horen een even aantal logische enen bevindt, de parity bit een logische 0 zal zijn ~\cite{Tijdscodering}. Het DCF77 blok converteert een gedigitaliseerde versie van het DCF77 signaal naar een tijdreferentie, waarna deze een autonome klok synchroniseert, welke zich ook binnen het DCF77 blok bevindt.

\begin{figure}[h!]
\center
\includegraphics[scale=1.9]{Figuren/DCF77/dcf77coding.png}
\caption{Codering van het dcf-signaal~\cite{Tijdscodering}}
\label{fig:dcfsignaal}
\end{figure}

\section{Specificaties}
In deze sectie worden de in- en uitgangen van de DCF-controller overzichtelijk weergeven. Doordat dit onderdeel aan het begin staat van het totale systeem, bevat dit blok enkel standaard ingangen en een ingang van buitenaf met het DCF77 signaal. De uitgangen uren en minuten worden doorgestuurd naar de main-controller. De clk van 1 Hz zal in verschillende onderdelen worden gebruikt, zowel binnen als buiten het DCF77 blok. De datum en het signaal dcf\_led zullen rechtstreeks op het LCD scherm worden weergegeven. Enkele signalen zijn in BCD (Binairy Coded Decimal). Meer informatie over BCD is te vinden op ~\cite{BCDinfo}.

\subsection{Ingangen}
Dit onderdeel maakt gebruik van de volgende ingangen: 
\begin{itemize}[nolistsep]
\item De 32 kHz systeemklok, een standaard input.
\item Het 'active high' resetsignaal, een standaard input.
\item Het gedigitaliseerde DCF77 signaal, bestaande uit korte en lange pulsen.
\end{itemize}
\noindent

\subsection{Uitgangen}
Dit onderdeel genereert de volgende uitgangen:
\begin{itemize}[nolistsep]
\item Een kloksignaal met een frequentie van 1 Hz, welke gebruikt kan worden om secondes te tellen.
\item Het debug signaal dcf\_led, wat een seconde lang hoog is na ontvangst van een puls van het DCF77 signaal.
\item Uren; de uren van de huidige tijd in een BCD vector van 6 bits.
\item Minuten; de minuten van de huidige tijd in een BCD vector van 7 bits.
\item Weekdag; de dag van de week, binair gecodeerd met maandag als 001.
\item Dag; de dag van de maand in een BCD vector van 6 bits.
\item Maand; het nummer van de maand in een BCD vector van 5 bits.
\item Jaar; de laatste twee cijfers van het jaartal in een BCD vector van 8 bits.
\end{itemize}

\subsection{Gedrag}
De DCF-controller heeft als belangrijkste gedragsfunctie om de tijd en datum door te geven. Al deze data moet zo vaak mogelijk worden gesynchroniseerd met een inkoment DCF-signaal. Dit signaal moet vanaf een antenne gedecodeerd worden en vervolgens moet het signaal via parity bits gecontroleerd worden. Pas als hieraan voldaan is moet het signaal in een register worden geschreven en doorgestuurd naar de andere onderdelen van de chip. Mocht het DCF-signaal signaal niet of slecht worden ontvangen dan moet de tijd instantaan en individueel door kunnen lopen. Indien het DCF-signaal weer terugkomt moet de controller dit automatisch oppakken. Om aan te geven of er een  DCF signaal is, moet er een zogenaamd debug lampje gaan branden als dit gebeurt. In de 60e seconde zal dit signaal laag moeten zijn, doordat dan geen bit doorgegeven wordt. \\
De huidige, indien mogelijk met het DCF77 signaal gesynchroniseerde, tijd moet worden gegeven in uren (in een BCD gecodeerd signaal van 5-bits) en minuten (in een BCD gecodeerd signaal van 6-bits). De datum, afkomstig uit het DCF77 signaal dient na synchronisatie te worden bewaard totdat er opnieuw met het DCF77 signaal wordt gesynchroniseerd. De datum wordt gegeven in vier vectoren:
\begin{itemize}[nolistsep]
\item De dag van de week wordt gegeven door drie bits, waarbij maandag is gecodeerd als 001, dinsdag als 010, enz.
\item De dag van de maand, gegeven als BCD gecodeerd signaal van 6-bits.
\item Het nummer van de maand wordt gegeven als BCD gecodeerd 5-bits signaal.
\item De laatste twee cijfers van het huidige jaartal worden bovendien gegeven door een 8-bits BCD gecodeerd signaal.
\end{itemize}
\vspace{0.3cm}
Daarnaast moet de DCF-controller ook een 1 Hz kloksignaal via een klokdeler naar alle andere onderdelen kunnen sturen.
Als laatste moeten alle subblokken, inclusief registers, van de DCF-controller bij een 'active high' van een resetknop gereset worden.