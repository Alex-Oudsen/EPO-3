\chapter{Introductie}
Epo 3 staat in het teken van het ontwerpen van een chip. Wat voor product er ontworpen gaat worden ligt aan de projectgroep. Het bedenken van het ontwerp is de eerste stap in het ontwerpproces, bij deze stap moet er al rekening gehouden met de randvoorwaarden die aan het project gesteld worden, zoals het aantal beschikbare transistoren op de chip en de beschikbare tijd.\\
Er is besloten om een wake-up light te maken. De belangrijkste functie is dat het licht 15 minuten voor de alarmtijd langzaam aan begint te gaan, totdat de lamp op de alarmtijd op volle sterkte brandt. Daarnaast zullen er nog een paar functies toegevoegd worden. Het DCF-signaal zal opgevangen worden voor de actuele datum en tijd, dit zal op een LCD-scherm worden laten zien. Door middel van vijf knoppen kan de wekker bediend worden. De alarmtijd kan ingesteld worden en de gebruiker kan aangeven of het licht en geluid aan moeten gaan als de gebruiker gewekt wil worden. Op de LCD zal ook te zien zijn of er iets aangepast wordt. De ingangs- en uitgangssignalen en het gedrag moeten geformuleerd worden als specificaties.\\
Er wordt structuur aangebracht in het systeem door het systeem op te delen in een paar grote blokken, deze blokken kunnen dan over de acht projectleden verdeeld worden. Allereerst moeten er van de afzonderlijke subsystemen specificaties opgesteld worden, zodat de blokken op elkaar afgestemd kunnen worden. Vervolgens moet van elk blok \'e\'en of meer FSM's gemaakt worden waarna er een code geschreven kan worden. De geschreven code moet gesimuleerd en gesyntetiseerd worden. Als aan het eind van het project van het hele systeem een lay-out gemaakt is, kan het systeem op een chip gezet worden.
