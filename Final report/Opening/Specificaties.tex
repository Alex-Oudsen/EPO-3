\chapter{Ontwerp specificatie}
% korte beschrijving wat de schakeling moet doen
Voor ons project ontwerpen we een klok, gesynchroniseerd met DCF77, weergeven op lcd met een wake-up alarm. De tijd, die intern wordt bijgehouden, zal worden gesynchroniseerd met een zogenaamd DCF signaal. De wekker zal bediend worden door middel van een menu. Dit menu wordt aangestuurd op basis van 4 knoppen. In dit menu moet de wekkertijd ingesteld worden. Ook moet de wekker en het wekkergeluid aan en uit gezet kunnen worden. Een vijfde knop is de uitknop voor als de wekker gaat en uitgezet moet worden. De visualisatie van dit menu zal op een LCD weergegeven worden. Als men zich niet in het menu bevindt, zal men alle data verdeeld over het scherm zien. Deze data bestaat uit de actuele tijd, de wekkertijd, de datum en de weekdag. Daarnaast zal op het LCD-scherm weergegeven worden of de wekker en het geluid aan staan. Met het knipperen van scheidingsteken tussen uren en minuten zal het passeren van seconden aangegeven worden.\\

% randvoorwaarden vastellen 
Het systeem zal enkele reandvoorwaarden hebben. Zo zal het een algemene reset moeten bevatten. Als gevolg van het indrukken van een resetknop zullen alle opgeslagen waarden en counters op 'nul' worden gezet. Ook zullen alle signalen 'active high' moeten zijn. De implementatie van het totale ontwerp moet op een chip-oppervlak van circa 0.4 cm\textsuperscript{2}, oftewel 40.000 transistor-paren. Dit in de vorm van twee opeengestapelde bond_bars, zoals aangelevert door de TU Delft. De chip beschikt over 32 pins voor i/o poorten. Daar zitten niet de voedingspinnen bij, deze worden apart aangesloten. Voor de FSM’s (Finite State Machine) mogen alleen die van het Moore-type gebruikt worden. Als de schakeling geactiveerd wordt moeten alle FSM’s in hun begintoestand komen door middel van een reset signaal. Voor de opwekking van het kloksignaal kan gebruik gemaakt worden van een kristal van 6.144 MHz of 32 kHz. Het streven is om zo weinig mogelijk componenten extern te gebruiken. De dissipatie van de chip dient echter ook beperkt te zijn. Dit geeft een compromis voor de maximale stroom die de elektronica mag dissiperen voor de aansturing van externe componenten, zoals LEDs. De voedingsspanning van het IC bedraagt 5 Volt. Het IC wordt gemaakt in een semi-custom CMOS proces. ~\cite{handleiding}\\

Het systeem zal de volgende ingangen hebben:
\begin{itemize}[nolistsep]
\item	DCF-signaal
\item	36kHz klok
\item	Reset-knop
\item	4 menu-knoppen
\item  1 uit-knop
\end{itemize}

\noindent
\\
Onze chip zal over de volgende uitgangen beschikken:
\begin{itemize}[nolistsep]
\item	LED, 1 bit om de led aan te sturen
\item	Sound, 1 bit om de buzzer aan te sturen
\item	LCD, een 7 bits vector om het scherm aan te sturen via een microcontroller
\item  Clk\_out, 1 bit ter aansturing van de microcontroller
\end{itemize}