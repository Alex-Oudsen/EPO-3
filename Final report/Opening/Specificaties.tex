\chapter{Ontwerp specificatie}
Het systeem moet aan verschillende specificaties voldoen. Zo zal het een algemene reset moeten bevatten. Als gevolg van het indrukken van de resetknop zullen alle opgeslagen waarden en counters op 'nul' worden gezet. Ook zullen alle signalen 'active high' moeten zijn. De tijd, die intern wordt bijgehouden, zal worden gesynchroniseerd met een zogenaamd DCF signaal.\\
De wekker zal bediend worden door middel van een menu. Dit menu wordt aangestuurd op basis van 4 knoppen. In dit menu moet de wekkertijd ingesteld worden. Ook moet de wekker en het wekkergeluid aan en uit gezet kunnen worden. Een vijfde knop is de uitknop voor als de wekker gaat en uitgezet moet worden.\\
De visualisatie van dit menu zal op een LCD weergegeven worden. Als men zich niet in het menu bevindt, zal men alle data verdeeld over het scherm zien. Deze data bestaat uit de actuele tijd, de wekkertijd, de datum en de weekdag. Daarnaast zal op het LCD-scherm weergegeven worden of de wekker en het geluid aan staan. Met het knipperen van scheidingsteken tussen uren en minuten zal het passeren van seconden aangegeven worden.\\
Het systeem zal de volgende ingangen hebben:
\begin{itemize}[nolistsep]
\item	DCF-signaal
\item	36kHz klok
\item	Reset-knop
\item	4 menu-knoppen
\item 1 uit-knop
\end{itemize}

\noindent
\\
Onze chip zal over de volgende uitgangen beschikken:
\begin{itemize}[nolistsep]
\item	LED, 1 bit om de led aan te sturen
\item	Sound, 1 bit om de buzzer aan te sturen
\item	LCD, een 8 bits vector om het scherm aan te sturen via een microcontroller
\item  Clk\_out, 1 bit ter aansturing van de microcontroller
\end{itemize}
