\documentclass[a4paper, oneside, 10pt]{article}

\usepackage{verbatim}
\usepackage[hidelinks]{hyperref}
\usepackage[dutch]{babel}
\usepackage{xcolor}
\usepackage{graphicx}
%\usepackage[table]{xcolor}
\usepackage{pdfpages}
\usepackage{mathtools}
%\usepackage{hyperref}
\usepackage{cleveref}
\usepackage{longtable}			% Table is te groot
\usepackage{listings}			% Used to include VHDL-code and fragments
\usepackage[dutch]{babel}		% Dutch hyphenation patterns and dutch names 
\usepackage{soul}				% dingen doorstrepen
\usepackage[normalem]{ulem}				%dingen dooruniten
\usepackage{pslatex}			% Times, helvetica and courier
\usepackage[T1]{fontenc}		% Nicer font-encoding
\usepackage{hyperref}			% Gives clickable references in pdf-file
\usepackage{graphicx}			% Used to include .pdf, .jpg and .png-files
\usepackage{tabularx}			% Used for evenly spread tables
\usepackage{eso-pic}			% Absolute positioning, used for lines-to-track appendix and front- and backpage
\usepackage{datetime}			% Used for some data-references
\usepackage[font=small,format=plain,labelfont=bf,up,textfont=up]{caption}	% Nicer captions
\usepackage{nonfloat}			% Captions for non-floating figures and tables
\usepackage{nextpage}			% Advanced nextpage commands
\usepackage{keystroke}			% "real" keys
\usepackage[nottoc]{tocbibind}		% Include Bibliography in ToC
\usepackage{multirow}			% Span text over multiple rows
\usepackage{verbatim}			% For comment-environment
\usepackage[left=3.5cm, right=2.5cm]{geometry}
\usepackage{enumitem} % Mogelijkheid tot geen enters in itemize en enurate

\crefname{equation}{vergelijking}{vergelijkingen}
\crefname{table}{tabel}{tabellen}
\crefname{figure}{figuur}{figuren}


\definecolor{comment}{RGB}{0, 15 , 117}		%Kleur blauw defineren
\definecolor{keyword}{RGB}{165, 42, 42}		%Kleur rood defineren
\definecolor{STD}{RGB}{46, 139, 87}			%Kleur groen defineren
\lstdefinelanguage{VHDL}{
  morekeywords=[1]{ 		%Defineren van keywords die blauw worden
    library,use,all,entity,is,port,in,out,end,architecture,of,
    begin,and,or,not,downto,ALL,signal,type,case,if,elsif,for,when,array,
    others,loop,process,to
  },
  morekeywords=[2]{			%Defineren van keyword die groen worden
    STD_LOGIC_VECTOR,STD_LOGIC,STD_LOGIC_1164,
    NUMERIC_STD,STD_LOGIC_ARITH,STD_LOGIC_UNSIGNED,std_logic_vector,unsigned,
    std_logic
  },
  morecomment=[l]--
}
\lstdefinestyle{vhdl}{
  language     = VHDL,
  basicstyle   = \footnotesize \ttfamily,
  keywordstyle = [1]\color{keyword}\bfseries, %Keywords kleuren
  keywordstyle = [2]\color{STD}\bfseries,	%Keywords kleuren
  commentstyle = \color{comment}, % Commits kleuren
  numbers=left,					  % Regel nummering
  breaklines=true,                % sets automatic line breaking
  tabsize=4                       % sets default tabsize to 4 spaces
}


\title{\textbf{States controller met alle uitgangen gedefineerd}}	

\author{
Gemaakt door leden van projectgroep A1\\
\begin{tabular}{c | l}
Rens Hamburger & 4292936 \\
\end{tabular}
}

\date{\today\\ Version 1.0} 

\begin{document}

\maketitle

\newpage
\section{Entity}
\scriptsize 
 \lstinputlisting [style= VHDL]{menu-ent.vhd}
 \normalsize
\section{Ingangen en uitgangen}
\begin{tabular}{|l|p{10cm}|}
\hline
wekdata & De huidige info over de wekker instellingen uit geheugen \newline
wekdata[5 down to 0] daarin staan de minuten \newline
wekdata[10 down to 6] daarin staan de uren \newline
wekdata[11] geluid bit \newline
wekdata[12] led bit \newline
wekdata[13] wekker bit (Of de wekker uberhaupt aan is of niet) \\ \hline
wekker & Bevat nieuwe informatie over de wekker \newline
wekdata[5 down to 0] daarin staan de minuten \newline
wekdata[10 down to 6] daarin staan de uren \newline
wekdata[11] geluid bit \newline
wekdata[12] led bit \newline
wekdata[13] wekker bit (Of de wekker uberhaupt aan is of niet) \\ \hline
toggle & Deze bit verandert bij veranderingen in de uitgang van wekkerdata \\ \hline
menu & Deze geeft door aan de in welke state we zitten aan de lcd module \newline
000 : Het normale scherm weergeven met alarm en wekkertijd weergave state: Rust,Wekkertijd \newline
001 : Uren aanpassen \newline
010 : Minuten aanpassen \newline
011 : Led aanpassen \newline
100 : Geluid aanpassen \\ \hline
\end{tabular}
\section{States}
\begin{longtable}{|l| p{10cm} |}
\hline
Rust, Reset &
enable ='0' \newline
wekker=wekdata \newline
toggle=niet toggle \newline
menu='000' \\ \hline
Wekker toggle &
enable ='1' \newline
wekker[12 down to 0]=wekdata[12 down to 0] \newline
wekker[13]=niet wekdata[13] \newline
toggle = niet toggle \newline
menu = menu \\ \hline
Wekkertijd &
enable ='0' \newline
wekker=wekdata \newline
menu = 000 \newline
toggle = niet toggle \\ \hline
Led &
enable ='0' \newline
wekker=wekdata \newline
menu = 011 \newline
toggle = niet toggle \\ \hline
Led toggle &
enable ='1' \newline
wekker[11 down to 0]=wekdata[11 down to 0]\newline
wekker[12] = niet wekdata[12] \newline
wekker[13] = wekdata[13] \newline
menu = menu \newline
toggle = niet toggle \\ \hline
Geluid & 
enable ='0' \newline
wekker=wekdata of een mooi getal code \newline
menu = 100 \newline
toggle = niet toggle \\ \hline
Geluid toggle &
enable ='1' \newline
wekker[10 down to 0]=wekdata[10 down to 0] \newline
wekker[11] = niet wekdata[11] \newline
wekker[13 downto 12] = wekdata[13 downto 12] \newline
menu = menu \newline
toggle=niet toggle \\ \hline
Uren instellen &
enable ='0' \newline
wekker=wekdata \newline
menu = 001 \newline
toggle =niet toggle \\ \hline
Uren plus &
enable ='1' \newline
wekker=wekker+1 \newline
menu =menu \newline
toggle = niet toggle \\ \hline
Uren min &
enable ='1' \newline
wekker=wekdata-1 \newline
menu \newline
toggle = niet toggle \\ \hline
Minuten instellen &
enable ='0' \newline
wekker=wekdata \newline
menu = 010 \newline
toggle =niet toggle \\ \hline
Minuten plus &
enable ='1' \newline
Wektijd=wekdata+1 \newline
menu = menu \newline
toggle = niet toggle \\ \hline
Minuten min &
enable ='1' \newline
wekker=wekdata-1 \newline
menu = menu \newline
toggle = niet toggle \\ \hline
\end{longtable}


\end{document}