\chapter{Main controller}
\section{Inleiding}
De main controller bevat de interface van de wekker. Deze zorgt er voor dat een wekker ingesteld kan worden, aangepast kan worden en uitgezet kan worden. Belangrijk aan elke interface is, dat deze gebruiksvriendelijk is. Dit kan onder andere bereikt worden door een optimum voor het aantal knoppen te bepalen. Te veel knoppen, en de gebruiker weet niet welke knop wat doet, te weinig knoppen, en de gebruiker moet navigeren door nodeloos ingewikkeld menu. \\
Daarnaast is er nog een beperkende factor: het aantal pinnen op de chip. \\
Al deze informatie samengenomen, is besloten dat 4 knoppen voor de interface het meest gebruiksvriendelijke resultaat oplevert. Daarnaast is er nog een knop die slechts gebruikt wordt om een afgaand alarm uit te zetten. \\
De controller stuurt een hoop dingen aan, en van te voren was al geanticipeerd dat dit hierdoor een van de grootste onderdelen van de chip zou kunnen worden.

\section{Specificaties}
\subsection{Ingangen}
\begin{itemize}[nolistsep]
\item Klok, dit is een standaard input;
\item Reset, ook dit is een standaard input;
\item Knoppen, dit zijn de 4 knoppen die (nadat ze gebufferd zijn) onderdeel zijn van de interface.
\begin{itemize}[nolistsep]
\item knoppen[0] = menu
\item knoppen[1] = set 
\item knoppen[2] = up
\item knoppen[3] = down\\
\end{itemize}
\end{itemize}



\subsection{Uitgangen}
\begin{itemize}[nolistsep]
\item Wekker, dit is de tijd dat de wekker af moet gaan en de wekkerdata, dus of het licht en geluid aan staan, en of de wekker uberhaupt aanstaat;
\item Menu-state, dit is de staat in welke de FSM zich op het moment bevindt. Deze informatie wordt doorgevoerd naar het LCD-scherm om zo te kunnen zien waar in het menu men zit.\\
\end{itemize}
In \cref{tab:uitgangen_controller} wat voor informatie te vinden is in de uitgangen van de controller.
\begin{table}[ht!]
\begin{tabular}{|l|p{10cm}|}
\hline
Uitgang & Informatie over wat in de uitgang te vinden is \\ \hline
wekdata & De huidige info over de wekker instellingen uit geheugen \newline
wekdata[5 down to 0] daarin staan de minuten \newline
wekdata[10 down to 6] daarin staan de uren \newline
wekdata[11] geluid bit \newline
wekdata[12] led bit \newline
wekdata[13] wekker bit (Of de wekker uberhaupt aan is of niet) \\ \hline
menu & Deze geeft door aan de in welke state we zitten aan de lcd module \newline
000 : Het normale scherm weergeven met alarm en wekkertijd weergave state: Rust,Wekkertijd \newline
001 : Uren aanpassen \newline
010 : Minuten aanpassen \newline
011 : Led aanpassen \newline
100 : Geluid aanpassen \\ \hline
\end{tabular}
\caption{Uitgangen van de controller}
\label{tab:uitgangen_controller}
\end{table}
\newpage
\subsection{Gedrag}
Om te beginnen moet de tijd waarop de wekker af moet gaan ingesteld kunnen worden. Dit wordt gedaan door eerst de huidige wekkertijd weer te geven, vervolgens het uur waarop gewekt moet worden te wijzigen en daarna de minuut. Hierna wordt de huidige tijd weer weergegeven. \\
Daarnaast is een vereiste dat de led uitgezet moet kunnen worden. Afhankelijk van een instelling moet het wake-up-light gedeelte wel of niet aangaan. Hetzelfde geld voor het geluid. \\
Dit alles moet zo gebruiksvriendelijk mogelijk gebeuren.

\section{Functionaliteit}

\subsection{FSM}
Wij gebruiken twee fsm een voor de buffer en een voor de menu.
\subsubsection{Buffer}
\subsubsection{Menu}

\subsection{VHDL code}
De code voor de controller van de wekker is te vinden in \cref{Ap:code_controller}. Voor de overzicht en het modular opbouwen is de code in vier blokken geschreven.
\begin{itemize}[nolistsep]
\item De top entity met de port map \cref{code:controller_ent,code:controller_beh}.
\item De menu opzichzelf hierin zit het echt logica in verwerkt te vinden in \cref{code:menu_ent,code:menu_beh}.
\item Het gebruikte geheugen element voor de opslag van 14 bits te vinden in \cref{code:geheugen_ent,code:geheugen_beh}.
\item De gebruikte buffer is te vinden in \cref{code:buffer_ent,code:buffer_ent}. De buffer regelt het ingangssignaal, en zorgt ervoor dat er maar 1 klokperiode lang een hoog signaal gelezen word
\end{itemize}
Voor het testen van de code zijn er testbenches gemaakt welke te vind zijn in \cref{code:tb_controller,code:tb_menu,code:tb_geheugen,code:tb_buffer}
\section{Resultaten}
Als eerste is een test gedaan door de VHDL code te simuleren met \emph{Modelsim}. Toen bleek dat dit een goed resultaat afleverde, is de code gesynthetiseerd, en is daarmee nog een simulatie gedaan. Toen bleek dat ook dit een goed resultaat afleverde, is de VHDL code geextraheerd, en deze code is opnieuw gesimuleerd. Vervolgens is nog een switch-level simulatie uitgevoerd, en hierna werd de code stabiel verklaard.

\subsection{Simulatie}

