                                                                                                                                                                                                                                                                                                                                                                  \chapter{Introduction}
Epo 3 staat in het teken van het ontwerpen van een chip. Wat voor product er ontworpen gaat worden ligt aan de projectgroep. Het bedenken van het ontwerp is de eerste stap in het ontwerpproces, bij deze stap moet er al rekening gehouden met de randvoorwaarden die aan het project gesteld worden.\\
Er is besloten om een wake-up light te maken. De belangrijkste functie is dat het licht 15 minuten voor de alarmtijd langzaam aan begint te gaan, totdat de lamp op de alarmtijd op volle sterkte brand. Daarnaast zullen er nog een paar functies toegevoegd worden. Het DCF-signaal zal opgevangen worden voor de actuele datum en tijd, dit zal op een LCD-scherm worden laten zien. Op de LCD zal ook de alarmtijd worden laten zien, die ingesteld kan worden. Al deze functies en de ingangs- en uitgangssignalen moeten geformuleerd worden als specificaties voordat er verder gegaan kan worden.\\
Er wordt structuur aangebracht in het systeem door het systeem op te delen in een paar grote blokken, deze blokken kunnen dan over de acht projectleden verdeeld worden. Van elk blok moeten \'e\'en of meer FSM's gemaakt worden waarna er een code geschreven kan worden. De geschreven code moet gesimuleerd en gesyntetiseerd worden. Als aan het eind van het project van het hele systeem een lay-out gemaakt is, kan het systeem op een chip gezet worden.
