\chapter{Voortgang van het project}

\section{Inleiding}
Bij dit project zijn er vaak weinig resultaten, totdat het bijna afgelopen is. 
Dit is een van de redenen dat voor een wake-up light gekozen is.
Een wekker zelf is relatief makkelijk te maken. 
Er zijn echter ook een hele hoop extra features die in een wekker ge\¨{i}mplementeerd kunnen worden. 
Op deze manier is dus een werkend resultaat relatief snel geproduceerd, en kunnen daarna naar gelang extra toepassingen toegevoegd worden. 
Dit is goed voor het moreel in de groep, aangezien een werkend product al heel snel gerealiseerd is.
Hierdoor is er ook meer aansporing om meer toepassingen te implementeren, omdat er al een werkend geheel is. 
In het ergste geval is er geen extra feature.
Daarnaast, als uiteindelijk bleek dat de planning te krap was, kunnen er features geschrapt worden, en is er nog steeds een werkend product. 
Onder andere dit maakt een wake-up light zeer aantrekkelijk om te maken.

\section{Werkverdeling}
De eerste twee weken werd er gewerkt aan een module-opdracht, in de vorm van een opwarmertje


\section{Samenwerking binnen de groep}

\section{Afspraken binnen de groep}
