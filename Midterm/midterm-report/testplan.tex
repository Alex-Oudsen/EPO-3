\chapter{Plan for testing the chip}
%Moet het gaan over het testen van de daadwerkelijke gemaakte chip in Q4 of moet het gaan over het testen op de fpga
Voor het testen zijn een aantal momenten in het proces waarop getest wordt. Zo wordt elk module getest in een simulatie in Modelsim. Hieruit kan opgemaakt worden wat het verwachte gedrag is. Maar een simulatie is niet alles. Daarom kan een module ook nog getest worden door middel van een FPGA te programmeren. De uiteindelijke chip zal getest worden met een logic analyzer en natuurlijk door te kijken of de chip de gewenste output geeft.
\section{FPGA bord}
Het bord dat gebruikt kan worden is een Altera FPGA bord. Dit bord komt met eigen software genaamt Quartus. Deze software kan gebruikt worden om de gemaakte VHDL code om te zetten in een bitstream file en vervolgens het FPGA bord te programmeren. Door de VHDL code op een FPGA te programmeren kan worden geverifieerd of de code het gedrag vertoont wat verwacht wordt. Door simulatie is dit namelijk niet altijd helemaal te zien. Mocht op de FPGA een fout ontdekt worden, dan zal de code hierop aangepast worden en zal de code opnieuw gesimuleerd worden.
\section{Logic Analyzer}
De gemaakte chip zal in Q4 worden getest. De chip zal eerst op een logic analyzer worden aangesloten. De analyzer die gebruikt zal worden is een LA-5580.
