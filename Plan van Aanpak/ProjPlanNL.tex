%=========================================================================
\documentclass[11pt,twoside,a4paper]{article}
\usepackage[dutch]{babel}
\usepackage{a4wide,times}
\usepackage{graphicx}
\title{Titel van het projectplan}
\author{Groep A1}
\date{30 November 2014}
\begin{document}
\maketitle
\thispagestyle{empty}
\vspace{30 mm}
\begin{center}
\Large \bf 
Samenvatting
\end{center}
Dit document beschrijft een indeling voor een projectplan (plan van aanpak) voor de groepsopdracht binnen het EPO-3 project. Het kan worden gebruikt als een sjabloon bij het schrijven van zo'n projectplan.  Zie ook hoofdstuk 4 en 5 uit ''Project Management'' van Roel Grit.
\clearpage

\tableofcontents
\clearpage

\section{Achtergrond}
Vertel iets over de achtergrond (context) waarbinnen het project zich afspeelt.
\section{Projectopdracht}
Wat zijn de doelstellingen van het project en wat is het eindresultaat ?  Richt je daarbij niet op de leerdoelen van het vak, maar op het product dat je levert. 
\section{Projectactiviteiten}
Wat zijn de verschillende activiteiten die moeten worden uitgevoerd om het eindresultaat te bereiken ?   Probeer - mede aan de hand van de ervaring met de FSM en de moduleopdracht - zo goed mogelijk in te schatten welke stappen hiervoor moeten worden gezet.
\section{Randvoorwaarden}
Wat zijn de randvoorwaarden waarmee tijdens het realiseren van het project rekening moet worden gehouden ?  Denk aan beschikbare tijd, mankracht en hulpmiddelen.
\section{Producten}
De activiteiten binnen het project resulteren in een aantal producten. Welke zijn dit ? Denk daarbij ook aan tussenrapporten en andere tussenresultaten.
\section{Kwaliteit}
Hoe kun je ervoor zorgen dat de kwaliteit van het eindproduct gewaarborgd wordt ? Welke hulpmiddelen zijn hiervoor te gebruiken ?  Denk hierbij ook aan tussentijdse kwaliteitscontrole.
\section{Projectorganisatie}
Wie doet er mee en hoe wordt er samengewerkt ?  Hoe wordt er gecoördineerd ?  Hoe vaak wordt er vergaderd ?  Welke afspraken worden er verder gemaakt voor communicatie en documentatie ?
\section{Planning}
Als je weet welke activiteiten er worden uitgevoerd, welke producten (milestones) er worden geleverd, en wie deelnemen aan het project, dan kan er een planning worden gemaakt. Hiervoor kan eventueel software zoals MicroSoft Project of GanttProject gebruikt worden.  De volgende overzichten zijn bijvoorbeeld gegenereerd m.b.v. het gratis te downloaden pakket GanttProject.

\begin{figure}[h!]
\includegraphics[width=\textwidth,height=\textheight,keepaspectratio,page=4]{planning}
\caption{Strokenplanning voor het project}
\label{strokenplan}
\end{figure}

\section{Risico's}
Wat zijn de gevaren die het succes het van het project bedreigen ? Wat doe je om de risico's op falen zo klein mogelijk te houden? Een van de mogelijke maatregelen is het werken met 'fall-back' scenario’s. Welke scenario's zijn dit dan?


\begin{thebibliography}{9}
\bibitem{labelboek1}
Auteur1, 
Boek1, 
Uitgeverij, 
jaar.
\bibitem{labelTitel}
Auteur1, 
Titel, 
Tijdschrift, 
Vol. 1, 
Nr. 1, 
jaar, 
pp. 12-15.
\bibitem{labelWeb}
Webpagina titel, 
http link, 
geraadpleegd op 19 sep. 2011.
\end{thebibliography}
\end{document}
